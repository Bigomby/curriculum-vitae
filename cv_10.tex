%%%%%%%%%%%%%%%%%%%%%%%%%%%%%%%%%%%%%%%%%
% Friggeri Resume/CV
% XeLaTeX Template
% Version 1.0 (5/5/13)
%
% This template has been downloaded from:
% http://www.LaTeXTemplates.com
%
% Original author:
% Adrien Friggeri (adrien@friggeri.net)
% https://github.com/afriggeri/CV
%
% License:
% CC BY-NC-SA 3.0 (http://creativecommons.org/licenses/by-nc-sa/3.0/)
%
% Important notes:
% This template needs to be compiled with XeLaTeX and the bibliography, if used,
% needs to be compiled with biber rather than bibtex.
%
%%%%%%%%%%%%%%%%%%%%%%%%%%%%%%%%%%%%%%%%%

\documentclass[hidelinks]{friggeri-cv} % Add 'print' as an option into the square bracket to remove colors from this template for printing

\begin{document}

\header{Diego } {Fernández Barrera}{Estudiante de Ingeniería de Telecomunicación} % Your name and current job title/field

%----------------------------------------------------------------------------------------
%	SIDEBAR SECTION
%----------------------------------------------------------------------------------------

\begin{aside} % In the aside, each new line forces a line break
\section{contacto}
Sevilla (Sevilla)
España
~
622 76 79 63
~
\href{mailto:bigomby@gmail.com}{bigomby@gmail.com}
\href{http://google.com/+DiegoFernandezBarrera}{+Diego Fernández}
\href{https://github.com/Bigomby}{GitHub: Bigomby}
\section{idiomas}
Español nativo
Inglés nivel medio
\section{lenguajes}
{Java, C, C++, bash, JavaScript, CSS3 \& HTML5, MATLAB, \LaTeX}
\end{aside}

%----------------------------------------------------------------------------------------
%	EDUCATION SECTION
%----------------------------------------------------------------------------------------

\section{estudios}

\begin{entrylist}
%------------------------------------------------
\entry
{2008--2015}
{Ingeniería {\normalfont de Telecomunicación}}
{Universidad de Sevilla}
{}
%------------------------------------------------
\end{entrylist}

%----------------------------------------------------------------------------------------
%	WORK EXPERIENCE SECTION
%----------------------------------------------------------------------------------------

\section{experiencia}

\begin{entrylist}
%------------------------------------------------
\entry
{Webs}
{}
{2013--2014}
{
\href{http://www.audiowho.com}{audiowho.com}

\href{http://grizzspain.com/}{grizzspain.com}

\href{http://centro-veterinario-la-loma.com/}{centro-veterinario-la-loma.com}
}
%------------------------------------------------
\entry
{Aplicaciones \\ Java}
{}
{2014}
{
\href{http://bigomby.github.io/pushtraps/}{Pushtraps}
}

\entry
{Aplicaciones \\ Android}
{}
{En Desarrollo}
{
\href{http://bigomby.github.io/mesas-ave/}{Comparte Mesas Ave}
}

\entry
{\LaTeX}
{}
{En Desarrollo}
{
\href{https://github.com/Bigomby/audiowho-novelas}{Proyecto de traducción de novelas y
maquetación en \LaTeX} 
}

%------------------------------------------------
\end{entrylist}

%----------------------------------------------------------------------------------------
%	AWARDS SECTION
%----------------------------------------------------------------------------------------

\section{conocimientos}

\begin{entrylist}
%------------------------------------------------
\entry
{Web}
{}
{}
{Desarrollo de webs en \textbf{HTML, CSS y JavaScript} usando diferentes frameworks como
\textbf{Bootstrap} o usando distintos CMS's como \textbf{WordPress} o \textbf{Joomla}.

También desarrollo de aplicaciones web usando tecnologías de \emph{backend} como
\textbf{Node.js} y en la actualidad aprendiendo tecnologías de frontend como
\textbf{Angular.js}.
}

\entry
{Desarrollo \\ Java}
{}
{}
{Programación de aplicaciones web usando Java en diferentes plataformas de escritorio.}

\entry
{Desarrollo \\ Android}
{}
{}
{Creación de programas para el Sistema Operativo móvil Android.}

\entry
{Administración \\ de sistemas \\ GNU/Linux}
{}
{}
{Conocimientos avanzados del Sistema Operativo GNU/Linux. Experiencia con 
distribuciones como Ubuntu, Debian, Fedora, ArchLinux, etc.}

\entry
{Administración \\ de sistemas \\ Windows}
{}
{}
{Conocimientos avanzados de Windows en diferentes versiones, desde XP hasta 8.1.}
\end{entrylist}

\begin{entrylist}
\entry
{Ofimática}
{}
{}
{Uso medio de diferentes suites como Microsoft Office, LibreOffice o Google Docs.
Tanto en procesadores de texto como en hojas de cálculos.}

\entry
{Electrónica/ \\ Robótica}
{}
{}
{Experiencia con microcontroladores como Arduino o Energía. Uso de sensores o comunicación
con sistemas XBee.}
\end{entrylist}


%----------------------------------------------------------------------------------------
%	INTERESTS SECTION
%----------------------------------------------------------------------------------------

\section{intereses}

\textbf{Profesional:} Me gustaría dedicarme al desarrollo y diseño web, programación, redes de datos, aplicaciones
móviles o a la administración de sistemas. Cualquier cosa relacionada con las telecomunicaciones
en general.

\textbf{Personal:} Me gusta el análisis de la seguridad en las tecnologías de la información
(Hacking, pentesting, criptoanálisis, etc.) y sobre todo me gusta arreglar cosas.



%----------------------------------------------------------------------------------------

\end{document}