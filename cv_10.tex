%%%%%%%%%%%%%%%%%%%%%%%%%%%%%%%%%%%%%%%%%
% Friggeri Resume/CV
% XeLaTeX Template
% Version 1.0 (5/5/13)
%
% This template has been downloaded from:
% http://www.LaTeXTemplates.com
%
% Original author:
% Adrien Friggeri (adrien@friggeri.net)
% https://github.com/afriggeri/CV
%
% License:
% CC BY-NC-SA 3.0 (http://creativecommons.org/licenses/by-nc-sa/3.0/)
%
% Important notes:
% This template needs to be compiled with XeLaTeX and the bibliography, if used,
% needs to be compiled with biber rather than bibtex.
%
%%%%%%%%%%%%%%%%%%%%%%%%%%%%%%%%%%%%%%%%%

\documentclass[hidelinks]{friggeri-cv} % Add 'print' as an option into the square bracket to remove colors from this template for printing
\usepackage{graphicx}
\usepackage[defaultsans]{opensans}
\usepackage[]{fontenc}
\usepackage[math-style=TeX,vargreek-shape=unicode]{unicode-math}
\usepackage{fontawesome}


\newfontfamily\bodyfont[]{TeX Gyre Heros}
\newfontfamily\thinfont[]{TeX Gyre Heros}
\newfontfamily\headingfont[]{TeX Gyre Heros}
\defaultfontfeatures{Mapping=tex-text}
\setmainfont[Mapping=tex-text, Color=textcolor]{TeX Gyre Heros}

\begin{document}

\header{Diego } {Fernández Barrera}{Arquitecto de Software}

%----------------------------------------------------------------------------------------
%	SIDEBAR SECTION
%----------------------------------------------------------------------------------------

\begin{aside} % In the aside, each new line forces a line break
 \includegraphics[width=\textwidth]{images/photo.png}
 \section{contacto}
 C/ Pasaje Oriente 1, 1º DCHA
 Morón de la Frontera (Sevilla)
 \mbox{}
 \faPhone\hspace{0.1em} 622 76 79 63
 \mbox{}
 \href{mailto:bigomby@gmail.com}{bigomby@gmail.com \hspace{0.1em} \faEnvelope }
 \href{http://linkd.in/1wwkBZE}{Diego Fernández \hspace{0.1em} \faLinkedin}
 \href{http://bit.ly/1ySLWLv}{Bigomby \hspace{0.1em} \faGithub}
 \section{idiomas}
 Español nativo
 Inglés nivel medio
 \section{lenguajes}
 {C, C++, Elixir, Go, HTML \& CSS3, JavaScript, \LaTeX, Rust}
\end{aside}

%----------------------------------------------------------------------------------------
%	EDUCATION SECTION
%----------------------------------------------------------------------------------------

\section{estudios}

\begin{entrylist}
 %------------------------------------------------
 \entry
 {2006--2008}
 {Bachillerato {\normalfont{} de Tecnología}}
 {I.E.S Fray Bartolomé de las Casas}
 {}
 \entry
 {2008--2017}
 {Ingeniería {\normalfont{} de Telecomunicación}}
 {Universidad de Sevilla}
 {}
 %------------------------------------------------
\end{entrylist}

%----------------------------------------------------------------------------------------
%	WORK EXPERIENCE SECTION
%----------------------------------------------------------------------------------------

\section{experiencia}

\begin{entrylist}
 \entry
 {2015}
 {Telefónica}
 {}
 {
  Beca Talentum de Telefónica en la que durante seis meses se trabajó como
  desarrollador Full Stack Javascript (Node.js + AngularJS) y sistemas
  integrados (Arduino).
 }
 \entry
 {2015}
 {redborder}
 {}
 {
  Prácticas curriculares y extracurriculares en redborder como desarrollador C.
 }
 \entry
 {2015--2017}
 {redborder}
 {}
 {
  Desarrollador de microservicios C/C++, NodeJS y Go.
 }
 \entry
 {2015--2017}
 {Oclose}
 {}
 {
  Arquitecto de Software con enfoque en microservicios, sistemas distribuidos
  y aplicaciones masivamente escalables.
 }
\end{entrylist}

%----------------------------------------------------------------------------------------
%	AWARDS SECTION
%----------------------------------------------------------------------------------------

\section{conocimientos}

\begin{entrylist}
 %------------------------------------------------
 \entry
 {Desarrollo \\ Web}
 {Frontend}
 {Nivel medio}
 {
  Desarrollo web frontend con \textbf{AngularJS} y actualmente con \textbf{Angular2}.
 }

 \entry
 {}
 {Backend}
 {Nivel alto}
 {
  Desarrollo de aplicaciones web con \textbf{NodeJS} y frameworks como
  \textbf{Express} o \textbf{Swagger} para APIs REST.\@
 }

 \entry
 {Desarrollo \\ Sistemas}
 {Microservicios}
 {Nivel avanzado}
 {
  Desarrollo de microservicios usando \textbf{NodeJS}, \textbf{Go} y \textbf{Elixir}
  para creación de sistemas distribuidos y altamente concurrentes y escalables.
 }

 \entry
 {}
 {Aplicaciones de alto rendimiento}
 {Nivel alto}
 {
  Desarrollo de aplicaciones de alto rendimiento usando lenguajes de bajo nivel
  como \textbf{C}, \textbf{C++} o \textbf{Rust}.
 }

 \entry
 {Internet \\ de las cosas}
 {Plataformas IoT}
 {Nivel avanzado}
 {
  Desarrollo de plataformas para servicios orientados al Internet de las Cosas.
  Conocimientos sobre \textit{brokers}, protocolos de mensajes, sistemas integrados,
  redes de sensores, microcontroladores, recolección de datos, etc.
 }
\end{entrylist}

%----------------------------------------------------------------------------------------

\end{document}
